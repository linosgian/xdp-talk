\documentclass{beamer}
\usepackage[T1]{fontenc}
\usepackage{alltt}
\usepackage{listings}
\usepackage{graphicx}
\makeatletter
\def\verbatim@font{\footnotesize\ttfamily}
\makeatother
\newcommand{\credit}[1]{\par\hfill \tiny Credit:~\itshape#1}


\usetheme{Warsaw}
\title[XDP]{How to build a DDoS Mitigation Pipeline using XDP}
\author{Linos Giannopoulos}
\date{December 20, 2018}
\begin{document}

\begin{frame}
\titlepage
\end{frame}

\begin{frame}{Agenda}
  \begin{itemize}
    \item cBPF
    \item eBPF
    \item XDP
    \item DDoS Mitigation Pipeline
  \end{itemize}
\end{frame}

\begin{frame}{cBPF}
Berkeley Packet Filter
\begin{itemize}
\item \textit{The BSD Packet Filter: A New Architecture for User-level Packet Capture}
\item Avoid context switches between kernel and user space
\item Filter packets as early as possible
\end{itemize}
\end{frame}

\begin{frame}[fragile]
  \frametitle{cBPF}
  \begin{verbatim}
# lgian[~] tcpdump -i eth1 ip and udp -d
(000) ldh      [12]
(001) jeq      #0x800      jt 2 jf 5
(002) ldb      [23]
(003) jeq      #0x11       jt 4 jf 5
(004) ret      #262144
(005) ret      #0

# lgian[~] strace -e trace=setsockopt tcpdump -i any tcp
...
setsockopt(3, SOL_SOCKET, SO_ATTACH_FILTER,
  {len=1, filter=0x7f0264d65000}, 16) = 0
setsockopt(3, SOL_SOCKET, SO_ATTACH_FILTER,
  {len=12, filter=0x55e81a792dd0}, 16) = 0
  \end{verbatim}
\end{frame}

\begin{frame}{eBPF}
  What changed
  \begin{itemize}
    \item Extended ISA for modern CPUs
    \item More registers (from 2 to 10)
    \item JIT compiler bytecode to native code
  \end{itemize}

  New concepts
  \begin{itemize}
    \item eBPF Maps shared between user and kernel space
    \item eBPF Verifier
  \end{itemize}
\end{frame}

\begin{frame}{eBPF Verifier}
  We essentially run userspace programs in the kernel. 
  \break

  How can we ensure an eBPF program won't cause a kernel panic or read/write arbitrary kernel memory?
  \break

  \pause

  Restrictions:

  \begin{itemize}
  \item 4096 instructions per program
  \item No loops
  \item No unreachable instructions
  \item Cannot read uninitialized registers
  \item Cannot read any arbitrary memory
  \item No out-of-bounds jumps
  \item etc.
  \end{itemize}
\end{frame}

\begin{frame}{eBPF}
  Some use cases:
  \begin{itemize}
    \item Tracing (kprobes/uprobes/tracepoints)\footnotemark
    \item Networking (tc/XDP)
    \item Security (seccomp/IDS/DDoS)
  \end{itemize}
  \footnotetext[1]{\url{https://www.youtube.com/watch?v=w8nFRoFJ6EQ}}
\end{frame}

% \begin{frame}[fragile]
%   \frametitle{eBPF Map Types}
%   \begin{itemize}
%     \item BPF\_MAP\_TYPE\_HASH: a hash table
%     \item BPF\_MAP\_TYPE\_ARRAY: an array map, optimized for fast lookup speeds, often used for counters
%     \item BPF\_MAP\_TYPE\_PERCPU\_ARRAY: a per-CPU array, used to implement histograms of latency
%     \item BPF\_MAP\_TYPE\_PERCPU\_HASH: a per-CPU hash table
%   \end{itemize}
% \end{frame}

\begin{frame}{XDP}
  eXpress Data Path
  \begin{itemize}
    \item High-performance packet processing
    \item Runs eBPF programs at the earliest possible* point (sk\_buffer metadata structure)
    \item In concert with the kernel (no userspace bypass or out-of-tree kernel modifications)
    \item Driver dependant (only several NIC drivers support it)
    \item Verdicts: aborted, drop, pass, tx
  \end{itemize}

  Use cases:
  \begin{itemize}
    \item Firewalling (Cillium\footnotemark[1])
    \item DDoS
    \item Load balancing (Katran\footnotemark[2])
    \item Monitoring
  \end{itemize}
  \footnotetext[1]{\tiny\url{https://cilium.io/}}
  \footnotetext[2]{\tiny\url{https://code.fb.com/open-source/open-sourcing-katran-a-scalable-network-load-balancer/}}
\end{frame}

\begin{frame}[fragile]
  \frametitle{XDP Example}
  \begin{verbatim}
    #include <linux/bpf.h>
    #include <linux/if_ether.h>
    #include <linux/ip.h>
    #define SEC(NAME) __attribute__((section(NAME), used))

    SEC("dropper")
    int xdp(struct xdp_md *ctx)
    {
        void *data = (void *)(long)ctx->data;
        void *data_end = (void *)(long)ctx->data_end;
        struct iphdr *ip = data + sizeof(struct ethhdr);

        // +1 is sizeof(struct iphdr)
        if (ip + 1 > data_end)
            return XDP_ABORTED;

        if(ip->saddr == 59017107){ // Binary format of "147.135.132.3"
            return XDP_DROP;
        }
        return XDP_PASS;
    }
  \end{verbatim}
\end{frame}

\begin{frame}{XDP Load Program}
  \begin{figure}
    \includegraphics[width=0.85\textwidth]{./xdp_workflow.png}
  \end{figure}
\end{frame}

\begin{frame}{XDP Load Program}
  \begin{itemize}
    \item Write C Program
    \item clang -O2 -target bpf -c xdp.c -o xdp.o
    \item ip link set dev eth0 xdpgeneric obj xdp.o sec <section-name>
  \end{itemize}
\end{frame}

\begin{frame}{XDP vs Everything}
  \Large{What about IPtables?}
\end{frame}

\begin{frame}{XDP vs Everything}
  Test bench
  \begin{itemize}
    \item 10GbE Intel NIC
    \item 14Mpps tiny UDP packets 
    \item Traffic is directed towards a single CPU
    \item Measure number of packets handled by the kernel on that CPU
    \item Randomized source IP/Port
  \end{itemize}

  \ \\
  \pause
  \textbf{\textit{How fast can we drop packets?}}
\end{frame}

\begin{frame}{XDP vs Everything}
  \begin{figure}
    \includegraphics[width=1.1\textwidth]{./no_xdp.png}
  \end{figure}
\end{frame}

\begin{frame}{XDP vs Everything}
  \begin{figure}
    \includegraphics[width=1.1\textwidth]{./with_xdp.png}
  \end{figure}
\end{frame}

\begin{frame}{Overview of Pipeline}
  \begin{itemize}
    \item Traffic sampling
    \item Traffic aggregation and analysis
    \item Attack reaction
    \item Attack mitigation
  \end{itemize}
\end{frame}

\begin{frame}{Traffic sampling}
  In contrast with an IDS, sampling instead of working on the whole traffic is crucial here.
  \begin{itemize}
    \item Some improvements are suggested in this area, but nothing concrete
    \item Use \textit{NFLOG}
    \item Userspace daemon and Sflow packets to central server
  \end{itemize}
\end{frame}

\begin{frame}{Traffic aggregation and analysis}
  How do we categorize traffic?
  \pause
  \begin{figure}
    \includegraphics[width=\linewidth]{./p0f.jpg}
    \credit{https://www.slideshare.net/InfoQ/xdp-in-practice-ddos-mitigation-cloudflare}
  \end{figure}
\end{frame}

\begin{frame}{Traffic aggregation and analysis}
  Traffic aggregated into groups
  \begin{itemize}
    \item TCP SYNs, TCP ACKs, UDP/DNS
    \item Destination IP/Port
    \item Known attack vectors and other heuristics
  \end{itemize}
\end{frame}

\begin{frame}[fragile]
\frametitle{p0f -> XDP}
\begin{verbatim}

static inline int match_p0f(void *data, void *data_end) {
    struct ethhdr *eth_hdr;
    struct iphdr *ip_hdr;
    struct tcphdr *tcp_hdr;
    u8 *tcp_opts;

    eth_hdr = (struct ethhdr *)data;
    if (eth_hdr + 1 > (struct ethhdr *)data_end)
        return XDP_ABORTED;
    if_not (eth_hdr->h_proto == htons(ETH_P_IP))
        return XDP_PASS;


\end{verbatim}
\end{frame}

\begin{frame}[fragile]
\frametitle{p0f -> XDP}
\begin{verbatim}
    ip_hdr = (struct iphdr *)(eth_hdr + 1);
    if (ip_hdr + 1 > (struct iphdr *)data_end)
        return XDP_ABORTED;
    if_not (ip_hdr->daddr == htonl(0x1020304))
        return XDP_PASS;
    if_not (ip_hdr->version == 4)
        return XDP_PASS;
    if_not (ip_hdr->ttl <= 64)
        return XDP_PASS;
    if_not (ip_hdr->ttl > 29)
        return XDP_PASS;
    if_not (ip_hdr->ihl == 5)
        return XDP_PASS;
    if_not ((ip_hdr->frag_off & IP_DF) != 0)
        return XDP_PASS;
    if_not ((ip_hdr->frag_off & IP_MBZ) == 0)
        return XDP_PASS;
\end{verbatim}
\end{frame}

\begin{frame}[fragile]
  \frametitle{p0f -> XDP}
  \begin{verbatim}
    tcp_hdr = (struct tcphdr*)((u8 *)ip_hdr +
    ip_hdr->ihl * 4);
    if (tcp_hdr + 1 > (struct tcphdr *)data_end)
        return XDP_ABORTED;
    if_not (tcp_hdr->dest == htons(1234))
        return XDP_PASS;
    if_not (tcp_hdr->doff == 10)
        return XDP_PASS;
    if_not ((htons(ip_hdr->tot_len) - (ip_hdr->ihl * 4) - (tcp_hdr->doff * 4)) == 0)
        return XDP_PASS;

    ...

    return XDP_DROP;
}
  \end{verbatim}
\end{frame}

\begin{frame}{Attack Reaction/Mitigation}
  \begin{itemize}
    \item Once we have a p0f signature of a specific attack
    \item Based on the signature generate an XDP program to drop the malicious traffic
    \item Push it to a distributed KV store
    \item Userspace daemon on each edge server that polls distributed KV store
    \item Update the XDP program deployed
  \end{itemize}
\end{frame}

\begin{frame}{Attack Reaction/Mitigation}
  \begin{figure}
    \includegraphics[width=0.9\textwidth]{./xdp_reaction.png}
    \credit{\url{https://netdevconf.org/2.1/papers/Gilberto_Bertin_XDP_in_practice.pdf}}
  \end{figure}
\end{frame}
\end{document}
